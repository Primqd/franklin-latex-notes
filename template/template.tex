\documentclass{article}

\usepackage{amsmath, amsthm, amssymb, amsfonts}
\usepackage{thmtools}
\usepackage{graphicx}
\usepackage{setspace}
\usepackage{geometry}
\usepackage{float}
\usepackage{hyperref}
\usepackage[utf8]{inputenc}
\usepackage[english]{babel}
\usepackage{framed}
\usepackage[dvipsnames]{xcolor}
\usepackage{tcolorbox}

\hypersetup{
    colorlinks=true,
    linkcolor=black,
    citecolor=blue,
    urlcolor=cyan
}

\graphicspath{{images/}}

\colorlet{LightGray}{White!90!Periwinkle}
\colorlet{LightOrange}{Orange!15}
\colorlet{LightGreen}{Green!15}
\colorlet{LightBlue}{Cyan!30}
\colorlet{LightRed}{Red!30}

\newcommand{\HRule}[1]{\rule{\linewidth}{#1}}

\declaretheoremstyle[
    name=Theorem,
    headpunct=\newline % Forces a newline after theorem title
]{thmsty}
\declaretheorem[style=thmsty,numberwithin=section]{theorem}
\tcolorboxenvironment{theorem}{colback=LightGray}

\declaretheoremstyle[
    name=Proposition,
    headpunct=\newline
]{prosty}
\declaretheorem[style=prosty,numberlike=theorem]{proposition}
\tcolorboxenvironment{proposition}{colback=LightOrange}

\declaretheoremstyle[
    name=Principle,
    headpunct=\newline
]{prcpsty}
\declaretheorem[style=prcpsty,numberlike=theorem]{principle}
\tcolorboxenvironment{principle}{colback=LightGreen}

\declaretheoremstyle[
    name=Example,
    headpunct=\newline
]{exsty}
\declaretheorem[style=exsty,numberlike=theorem]{example}
\tcolorboxenvironment{example}{colback=LightBlue}

\declaretheoremstyle[
    name=Definition,
    headpunct=\newline
]{defsty}
\declaretheorem[style=defsty,numberlike=theorem]{definition}
\tcolorboxenvironment{definition}{colback=LightRed}


\setstretch{1.2}
\geometry{
    textheight=9in,
    textwidth=5.5in,
    top=1in,
    headheight=12pt,
    headsep=25pt,
    footskip=30pt
}

% ------------------------------------------------------------------------------

\begin{document}

% ------------------------------------------------------------------------------
% Cover Page and ToC
% ------------------------------------------------------------------------------

\title{ \normalsize \textsc{}
		\\ [2.0cm]
		\HRule{1.5pt} \\
		\LARGE \textbf{\uppercase{Template Title}
		\HRule{2.0pt} \\ [0.6cm] \LARGE{Subtitle} \vspace*{10\baselineskip}}
		}
\date{}
\author{\textbf{Author} \\ 
		Who? \\
		Where? \\
		When?}

\maketitle
\newpage

\tableofcontents
\newpage

% ------------------------------------------------------------------------------

\section{Examples}

\begin{theorem}
    This is a theorem.
\end{theorem}

\begin{proposition}
    This is a proposition.
\end{proposition}

\begin{principle}
    This is a principle.
\end{principle}

\begin{example}[test] % new line
    this is an example
\end{example}

\begin{definition}[defintion]
    this defines something
\end{definition}

% Maybe I need to add one more part: Examples.
% Set style and colour later.

\subsection{Pictures}

\begin{figure}[htbp]
    \center
    \includegraphics[scale=0.06]{shitpost.png}
    \caption{Sydney, NSW}
\end{figure}

\subsection{Citation}

This is a citation\cite{ctan}.

\newpage

% ------------------------------------------------------------------------------
% Reference and Cited Works
% ------------------------------------------------------------------------------

\bibliographystyle{IEEEtran}
\bibliography{references.bib}

% ------------------------------------------------------------------------------

\end{document}