\documentclass[12pt, a4paper]{article}
% \usepackage{mathtools}
\usepackage{graphicx}
\usepackage{amsthm}
\usepackage{hyperref}
\usepackage{amssymb}
% \graphicspath{{images/}}

\hypersetup{
    colorlinks=true,
    linkcolor=blue,
    urlcolor=cyan
}

\title{Work and Energy Notes}
\author{Franklin Chen}
\date{14 October 2024}

\theoremstyle{definition}
\newtheorem{definition}{Definition}

\begin{document}
\maketitle
\newpage
% comment

\section{What is Work?}

\begin{definition}[Work]
Energy transferred to or from an object via the application of force along displacement. Work is equal to the dot product of force and displacement:
\[W = \vec{F} \cdot{} \vec{s} = ||F|| * ||s|| * \cos(\theta)\]

The SI unit for work is $\mathrm{Newtons \cdot{} Meters = Joules}$ (J).
\end{definition}
The second definition $W = ||F|| * ||s|| * \cos(\theta)$ can be obtained by the definition of the dot product, where $\theta$ represents the angle between the force and displacement vectors.

Geometrically, the defintion of work can be thought of as the component of the force vector that has the same direction as the displacement vector multiplied by the displacement vector.

\emph{No work does not imply no force or no displacement.} The definition of work suggests that if the entire force is perpendicular to the displacement, \emph{no work is done} (by the perpendicular component of the force)

Work can also be negative, depending on whether the component of the force points in the same or opposite direction as the displacement. For example, a weight lifter lowering a barbell at a constant velocity supplies force upward (to counteract the force of gravity), yet the net displacement is downward, causing negative work.

Note this definition of work only works for \emph{constant forces}. We can use intergration to find the net work for variable forces:

\[W_{ab} = \int_{a}^{b} \mathrm{\vec{F} \, \vec{ds}} = \int_{a}^{b} \mathrm{F\cos{\theta}\, ds}\]
\newpage

\section{Kinetic Energy}
In physics, whenever a net force performs work on an object, there is always a result from the effort; a change in the \emph{kinetic energy} of the object. The relationship that relates work to kinetic energy is known as the \emph{work-energy theorem.}\newline

Multiplying Newton's Second Law $\sum F = ma$ by displacement \emph{s} yields

\[(\sum F)s = mas\]

\emph{as} can be related to $v_i$ and $v_f$ via the equation $v_f^2 = v_i^2 + 2as$:

\[as = \frac{1}{2}(v_f^2 - v_i^2)\]

This definition of \emph{as} can be substituted into the first equation to yield the \emph{work-energy theorem}:

\begin{definition}[Work-Energy Theorem]
\[W = \frac{1}{2}mv_f^2 - \frac{1}{2}mv_i^2\]
When a net external force does work \emph{W} on an object, the work is equal to the difference between the final and initial kinetic energy. 
\end{definition}

As implied by the work-energy theorem, the quantity $\frac{1}{2}\mathrm{(mass)(velocity)^2}$ is known as \emph{kinetic energy}:

\begin{definition}[Kinetic Energy]
The kinetic energy \textbf{KE} of an object with mass $m$ and $v$ is given by
\[\mathrm{KE} = \frac{1}{2}mv^2\]
The SI unit for kinetic energy is the Joule.
\end{definition}

Note the work-energy theorem applies to \textbf{any} scenario where an object changes its velocity (and as such, its kinetic energy), as work \emph{must} be done to accelerate (or deccelerate) the object from initial speed $v_i$ to a final speed $v_f$. 

\newpage

\section{Gravitational Potential Energy}
\subsection{Work done by Gravity}
Using the equation $W = ||F|| * ||s|| * \cos(\theta)$, for an object dropping down due to gravity, as the only force acting on it is its weight $w = mg$ (assuming the objects are relatively close to earth),
its displacement is equal to $h_0 - h_f$, and both weight and displacement are in the same direction, its work is equal to

\[W_{grav} = mg(h_0 - h_f)\]

Note this equation is valid for \emph{any path} between the initial and final heights, \emph{regardless of any curvature in the path} (i.e. throwing the ball up).

\subsection{Gravitational Potential Energy}
An object may posess energy by virtue of its position relative to the earth: it may be able to change its potential energy into kinetic energy by dropping.
This "potential" energy is known as \emph{gravitational potential energy}.

\begin{definition}[Gravitational Potential Energy (PE)]
    The energy that an object of mass $m$ has by virtue of its position relative to the surface of the earth. That position is measured by the height $h$ of the object relative to an
    abritary zero value:

    \[\mathrm{PE} = mgh\]

    The SI unity for PE is the Joule.
\end{definition}

PE is derived from the difference between two potential energies: that is, $W_{grav} = mgh_0 - mgh_f$.
Note that the arbitary zero value can be defined anywhere, as long as both the initial and final heights are measured relative to that zero.



\end{document}