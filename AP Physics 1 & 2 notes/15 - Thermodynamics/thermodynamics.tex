\documentclass[12pt, a4paper]{article}
% \usepackage{mathtools}
\usepackage{graphicx}
\usepackage{amsthm}
\usepackage{hyperref}
\usepackage{amssymb}
% \graphicspath{{images/}}

\hypersetup{
    colorlinks=true,
    linkcolor=blue,
    urlcolor=cyan
}

\title{Thermodynamics}
\author{Franklin Chen}
\date{7 December 2024}

\theoremstyle{definition}
\newtheorem{definition}{Definition}

\begin{document}
\maketitle
\newpage
% comment

\tableofcontents

\section{Definition}
\begin{definition}[Thermodynamics]
    The branch of physics that analyzes the relation between heat and work. Often, both occur together.
\end{definition}

\begin{definition}[System, Surroundings]
    The system is the collection of objects on which attention is focused.
    The surroundings is everything else in the environment.
\end{definition}

\begin{definition}[Diathermal and Adiabatic Walls]
    Walls seperate the system from the surroundings.
    Diathermal walls permit heat to flow through them.
    Adibatic walls are perfectly insulating walls that perfectly prevent heat from flowing between the system and the surroundings.
\end{definition}

\begin{definition}[State of a System]
    A complete description of a system at a given time.
    Specified by variables for pressure, volume, temperature, and entropy for thermodynamics.
\end{definition}

\begin{definition}[Thermal Equilibrium]
    Two systems are in thermal equilibrium if there is no net flow of heat between them when they are brought into thermal contact.
\end{definition}

\section{The Zeroth Law of Thermodynamics}
\begin{definition}[The Zeroth Law of Thermodynamics]
    Two systems individually in thermal equilibrium with a third system are in thermal equilibrium with each other.
    The state of the third system is the same when it is in thermal equilibrium with either of the systems.
\end{definition}

The third system is often a thermometer; if two objects have the same temperature, then there are in thermal equilibrium.
In effect, the zeroeth law establishes temperature as the indicator of thermal equilibrium (same temperature = thermal equilibrium)
\textit{and implies that all parts of a system must be in thermal equilibrium if the system is to have one definable temperature.}


\section{The First Law of Thermodynamics}

\subsection{Thermal Processes}

\section{The Second Law of Thermodynamics}

\subsection{Carnot Engines}

\subsection{Entropy}

\section{The Third Law of Thermodynamics}



\end{document}